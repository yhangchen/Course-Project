\documentclass[conference,onecolumn,12pt]{IEEEtran}
\usepackage{graphicx,amsmath,amsfonts,amssymb,bm,hyperref,url,breakurl,epsfig,epsf,color,fullpage,MnSymbol,mathbbol,fmtcount,algorithmic,algorithm,semtrans,cite,multirow}
\usepackage{subfigure}
\usepackage{mathtools}
\DeclarePairedDelimiter{\floor}{\lfloor}{\rfloor}
\usepackage{palatino}
\usepackage{booktabs}

\usepackage{tikz}
\usepackage{pgfplots}
\usetikzlibrary{pgfplots.groupplots}
\usepackage{setspace}
\renewcommand{\baselinestretch}{1.5}
\usepackage{fancyhdr}

% \setcounter{secnumdepth}{4}

\usepackage{cite}
\usepackage{caption}
\usepackage[bottom,hang,flushmargin]{footmisc} 

\setlength{\captionmargin}{30pt}

\usepackage{hyperref}
\definecolor{darkred}{RGB}{150,0,0}
\definecolor{darkgreen}{RGB}{0,150,0}
\definecolor{darkblue}{RGB}{0,0,200}
\hypersetup{colorlinks=true, linkcolor=blue, citecolor=blue, urlcolor=blue}

%--------------
\newtheorem{theorem}{Theorem}[section]
\newtheorem{lemma}[theorem]{Lemma}
\newtheorem{corollary}[theorem]{Corollary}
\newtheorem{proposition}[theorem]{Proposition}
\newtheorem{definition}[theorem]{Definition}
\newtheorem{condition}[theorem]{Condition}
\newtheorem{conjecture}[theorem]{Conjecture}

\newtheorem{remark}[subsection]{Remark}
\newtheorem{remarks}[subsection]{Remarks}
\newtheorem{example}[subsection]{Example}

%% TT's definitions
\newcommand\tr{{{\operatorname{trace}}}}
\newcommand{\eps}{\varepsilon}


%%YP's macros
\newcommand{\yp}[1]{\textcolor{red}{ #1}}
\newcommand{\zeronorm}[1]{\left\|#1 \right\|_0}
\newcommand{\unorm}[1]{\left\|#1 \right\|_u}
\newcommand{\ynorm}[1]{\left\|#1 \right\|_{\bar{y}}}
\newcommand{\onetwonorm}[1]{\left\|#1\right\|_{1,2}}
\newcommand{\opnorm}[1]{\left\|#1\right\|}
\newcommand{\fronorm}[1]{\left\|#1\right\|_{F}}
\newcommand{\onenorm}[1]{\left\|#1\right\|_{\ell_1}}
\newcommand{\twonorm}[1]{\left\|#1\right\|}
\newcommand{\Dnorm}[1]{\left\|#1\right\|_{D}}
\newcommand{\oneinfnorm}[1]{\left\|#1\right\|_{1,\infty}}
\newcommand{\infnorm}[1]{\left\|#1\right\|_{\ell_\infty}}
\newcommand{\nucnorm}[1]{\left\|#1\right\|_*}
\newcommand{\abs}[1]{\left|#1\right|}
\newcommand{\avg}[1]{\left< #1 \right>}

\renewcommand{\d}{\mathrm{d}}

\newcommand{\cA}{\mathcal{A}}
\newcommand{\x}{\vct{x}}
\newcommand{\y}{\vct{y}}
\newcommand{\W}{\mtx{W}}
%--------------
\newcommand{\mapA}{\mathbb{A}}

% EJC's macros

\definecolor{emmanuel}{RGB}{255,127,0}
\newcommand{\ejc}[1]{\textcolor{emmanuel}{EJC: #1}}

\newcommand{\R}{\mathbb{R}}
\newcommand{\C}{\mathbb{C}}
\newcommand{\Z}{\mathbb{Z}}
\newcommand{\<}{\langle}
\renewcommand{\>}{\rangle}
\newcommand{\Var}{\textrm{Var}}
\newcommand{\sgn}[1]{\textrm{sgn}(#1)}
\renewcommand{\P}{\operatorname{\mathbb{P}}}
\newcommand{\E}{\operatorname{\mathbb{E}}}
\newcommand{\todo}[1]{{\bf TODO: #1}}
\newcommand{\e}{\mathrm{e}}
\renewcommand{\i}{\imath}

\newcommand{\vct}[1]{\bm{#1}}
\newcommand{\mtx}[1]{\bm{#1}}

% MS's macros
\newcommand{\trace}{\operatorname{Tr}}

\newcommand{\rank}{\operatorname{rank}}
\newcommand{\supp}[1]{\operatorname{supp}(#1)}
\newcommand{\restrict}[1]{\big\vert_{#1}}
\newcommand{\Id}{\text{\em I}}
\newcommand{\OpId}{\mathcal{I}}

\newcommand{\Real}{\operatorname{Re}}
\newcommand{\Imag}{\operatorname{Im}}

\newcommand{\piyp}{\pi'_1(\vct{y})}
\newcommand{\piy}{\pi_1(\vct{y})}
\newcommand{\piar}{\pi_1(\vct{a}_r)}
\newcommand{\piarp}{\pi'_1(\vct{a}_r)}


\newcommand{\MS}[1]{\textcolor{blue}{MS: #1}}
\newcommand{\MST}[1]{\textcolor{red}{#1}}
\newcommand{\XL}[1]{\textcolor{cyan}{XL: #1}}


\numberwithin{equation}{section} 

\def \endprf{\hfill {\vrule height6pt width6pt depth0pt}\medskip}
\newenvironment{proof}{\noindent {\bf Proof} }{\endprf\par}


\newcommand{\qed}{{\unskip\nobreak\hfil\penalty50\hskip2em\vadjust{}
           \nobreak\hfil$\Box$\parfillskip=0pt\finalhyphendemerits=0\par}}


\newcommand{\note}[1]{{\bf [{\em Note:} #1]}}

\newcommand{\iprod}[2]{\left\langle #1 , #2 \right\rangle}

\newcommand{\red}{\textcolor{red}}

\renewcommand{\algorithmicrequire}{\textbf{Input:}}
\renewcommand{\algorithmicensure}{\textbf{Output:}}

\begin{document}
\title{Phase Retrieval\\
{\footnotesize \textsuperscript{*}Project 4 on the Course ``Algorithms for Big Data Analysis".}
}

\author{\IEEEauthorblockN{1\textsuperscript{st} Chen Yihang}
\textit{Peking University}\\
1700010780}
\date{}

\maketitle
\thispagestyle{fancy} % IEEE模板在\maketitle后会自动声明\thispagestyle{plain},
% 导致第一页什么都没有。所以得把plain更改为fancy
\lhead{} % 页眉左,需要东西的话就在{}内添加
\chead{} % 页眉中
\rhead{} % 页眉右
\lfoot{} % 页眉左
\cfoot{} % 页眉中
\cfoot{\thepage} %页眉右,\thepage 表示当前页码
\renewcommand{\headrulewidth}{0pt} %改为0pt即可去掉页眉下面的横线
\renewcommand{\footrulewidth}{1pt} %改为0pt即可去掉页脚上面的横线
\pagestyle{fancy}
\cfoot{\thepage}


\begin{abstract}
  In this report, I implemented the PhaseLift \cite{candes2013phase} and Wirtinger flow \cite{candes2015phase} to solve the phase retrieval problem. We test the two algorithms on the coded diffraction model, and find out that Wirtinger flow is much more efficient. Besides, we also discuss the ``probability of successful retrieval'' on different $m/n$.
\end{abstract}


\tableofcontents
\newpage
\section{Problem setting}
One popular formulation of the phase retrieval problem is solving a system of quadratic equations in the form
\begin{align}
\label{quadeq}
y_r=\abs{\<\vct{a}_r, \vct{z}\>}^2,\quad r=1,2,\ldots,m,
\end{align}
where $z\in \mathbb{C}^n$ is the decision variable, $a_r\in \mathbb{C}^n$ are known sampling vectors, $<a_r,z>$ is the inner product between $a_r$ and $z$ in $\mathbb{C}^n$, $|a|$ is the magnitude of $a\in \mathbb{C}$, and $y\in \mathbb{R}$ are the observed measurements. This problem is a general instance of a nonconvex quadratic program (QP). In this report, I implemented the gradient method of Wirtinger flow \cite{candes2015phase} and PhaseLift \cite{candes2013phase}. In the following, I give a brief introduction to both of the algorithms.

\section{Wirtinger flow}
The Wirtinger flow includes two steps: (1) a careful initialization obtained by means of a spectral method, and (2) a series of updates refining this initial estimate by iteratively applying a novel update rule, much like in a gradient descent scheme. 
\subsection{Minimization of a non-convex objective}

Let $\ell(x,y)$ be a loss function measuring the misfit between both
its scalar arguments. If the loss function is non-negative and
vanishes only when $x = y$, then a solution to the generalized phase
retrieval problem \eqref{quadeq} is any solution to
\begin{equation}
\label{noncvx}
\text{minimize} \quad f(\vct{z}):=\frac{1}{2m}\sum_{r=1}^m
\ell\bigl(y_r, \abs{\vct{a}_r^*\vct{z}}^2\bigr), \quad \vct{z} \in \C^n.  
\end{equation}
Although one could study many loss functions, we shall focus in this
paper on the simple quadratic loss $\ell(x,y)=(x-y)^2$. Admittedly,
the formulation \eqref{noncvx} does not make the problem any easier
since the function $f$ is not convex. Minimizing non-convex
objectives, which may have very many stationary points, is known to be
NP-hard in general.  In fact, even establishing convergence to a local
minimum or stationary point can be quite challenging.

Our approach to \eqref{noncvx} is simply stated: start with an
initialization $\vct{z}_0$, and for $\tau = 0, 1, 2, \ldots$,
inductively define
\begin{equation}
\label{graddescent}
\vct{z}_{\tau+1}=\vct{z}_\tau-\frac{\mu_{\tau+1}}{\twonorm{\vct{z}_0}^2}\left(\frac{1}{m}\sum_{r=1}^m\left(\abs{\vct{a}_r^*\vct{z}}^2-y_r\right)(\vct{a}_r\vct{a}_r^*)\vct{z}\right):=\vct{z}_\tau-\frac{\mu_{\tau+1}}{\twonorm{\vct{z}_0}^2}\nabla
f(\vct{z}_\tau). 
\end{equation}
If the decision variable $\vct{z}$ and the sampling vectors were all
real valued, the term between parentheses would be the gradient of
$f$ divided by two, as our notation suggests. However, since $f(\vct{z})$ is a
mapping from $\C^n$ to $\R$, it is not holomorphic and hence not
complex-differentiable. However, this term can still be viewed as a
gradient based on Wirtinger derivatives. Hence, \eqref{graddescent} is a form of steepest descent and the parameter $\mu_{\tau+1}$ can be interpreted as a step size (note nonetheless that the effective step size is also inversely proportional to the magnitude of the initial guess).

\subsection{Initialization via a spectral method}
\label{Initschemes}

Our main result states that for a certain random model, if the
initialization $\vct{z}_0$ is sufficiently accurate, then the sequence
$\{\vct{z}_{\tau}\}$ will converge toward a solution to the
generalized phase problem \eqref{quadeq}.  In this paper, we propose
computing the initial guess $\vct{z}_0$ via a spectral method,
detailed in Algorithm \ref{QIINIT}. In words, $\vct{z}_0$ is the
leading eigenvector of the positive semidefinite Hermitian matrix
$\sum_r y_r \vct{a}_r \vct{a}_r^*$ constructed from the knowledge of
the sampling vectors and observations. (As usual, $\vct{a}_r^*$ is the
adjoint of $\vct{a}_r$.) Letting $\mtx{A}$ be the $m \times n$ matrix
whose $r$th row is $\vct{a}_r^*$ so that with obvious notation
$\vct{y} = |\mtx{A} \vct{x}|^2$, $\vct{z}_0$ is the leading
eigenvector of $\mtx{A}^* \, \operatorname{diag}\{\vct{y}\}\, \mtx{A}$
and can be computed via the power method by repeatedly applying
$\mtx{A}$, entrywise multiplication by $\vct{y}$ and $\mtx{A}^*$.
\begin{algorithm}[h]
  \caption{Wirtinger Flow: Initialization} 
\begin{algorithmic}
  \REQUIRE{Observations $\{y_r\} \in \R^m$.}
%
  \STATE Set 
\[ 
\lambda^2 = n \, \frac{\sum_{r}
    y_r}{\sum_r \twonorm{\vct{a}_r}^2}.
\]
% Issue of normalization
\STATE Set $\vct{z}_0$, normalized to $\|\vct{z}_0\| = \lambda$, to be
the eigenvector corresponding to the largest eigenvalue of
  \begin{align*}
  \mtx{Y}=\frac{1}{m}\sum_{r=1}^my_r\vct{a}_r\vct{a}_r^*.
  \end{align*}
  \ENSURE{Initial guess $\vct{z}_0$.}
\end{algorithmic}
\label{QIINIT}
\end{algorithm}

We consider a simple model where everything is
real valued, and in which the vectors $a_r$ are
i.i.d.~$\mathcal{N}(\vct{0},\mtx{I})$. Also without any loss in generality and to simplify exposition in this section we shall assume $\twonorm{\vct{x}}=1$.

    Let $\vct{x}$ be a solution to \eqref{quadeq} so that $y_r = |\<
\vct{a}_r, \vct{x}\>|^2$, and consider the initialization step first.
In the Gaussian model, a simple moment calculation gives
\[
\E \left[\frac{1}{m}\sum_{r=1}^my_r\vct{a}_r\vct{a}_r^*\right] =
\mtx{I} + 2 \vct{x}\vct{x}^*. 
\]
By the strong law of large numbers, the matrix $\mtx{Y}$ in Algorithm
\ref{QIINIT} is equal to the right-hand side in the limit of large
samples. Since any leading eigenvector of $\mtx{I} + 2
\vct{x}\vct{x}^*$ is of the form $\lambda \vct{x}$ for some scalar
$\lambda \in \R$, we see that the intialization step would recover
$\vct{x}$ perfectly, up to a global sign or phase factor, had we
infinitely many samples. Indeed, the chosen normalization would
guarantee that the recovered signal is of the form $\pm \vct{x}$.  As
an aside, we would like to note that the top two eigenvalues of
$\mtx{I} + 2 \vct{x}\vct{x}^*$ are well separated unless $\|\vct{x}\|$
is very small, and that their ratio is equal to $1 + 2
\|\vct{x}\|^2$.  Now with a finite amount of data, the leading
eigenvector of $\mtx{Y}$ will of course not be perfectly correlated
with $\vct{x}$ but we hope that it is sufficiently correlated to point
us in the right direction.



\section{PhaseLift Algorithm}
\subsection{Lifting}
Suppose we have $x_0 \in \C^n$ or $\C^{n_1 \times n_2}$ (or some
higher-dimensional version) about which we have quadratic measurements
of the form
\begin{equation}
  \mapA(x_0) = \{|\<a_k, x_0\>|^2 : k = 1, 2, \ldots, m\}.
\label{nonlinearmap}
\end{equation}

Phase retrieval is then the feasibility problem
\begin{equation}
\label{eq:retrieval}
  \begin{array}{ll}
    \text{find}   & \quad x\\ 
\text{obeying} & \quad  \mapA(x) = \mapA(x_0) := b.
  \end{array}
\end{equation}

As is well known, quadratic measurements can be lifted up and
interpreted as linear measurements about the rank-one matrix $X = x
x^*$. Indeed,
\[
|\<a_k, x\>|^2 = \trace(x^* a_k a_k^* x) = \trace(a_k a_k^* x x^*) :=
\trace(A_k X),
\]
where $A_k$ is the rank-one matrix $a_k a_k^*$. % For later reference
% it will be useful to denote 
% \begin{equation}
% A: = [a_1, a_2, \dots , a_m]^\ast.
% \label{framematrix}
% \end{equation}
In what follows, we
will let $\cA$ be the linear operator mapping positive semidefinite
matrices into $\{\trace(A_k X) : k = 1,\ldots, m\}$.  Hence, the phase
retrieval problem is equivalent to
\begin{equation}
\label{eq:rankmin}
  \begin{array}{ll}
    \text{find}   & \quad X\\ 
    \text{subject to} & \quad  \cA(X) = b\\
& \quad X \succeq 0\\
& \quad \rank(X) = 1
\end{array} 
\qquad \Leftrightarrow \qquad 
  \begin{array}{ll}
    \text{minimize}   & \quad \rank(X)\\ 
    \text{subject to} & \quad  \cA(X) = b\\
& \quad X \succeq 0. 
\end{array}
\end{equation}
Upon solving the left-hand side of \eqref{eq:rankmin}, we would
factorize the rank-one solution $X$ as $x x^*$, hence finding
solutions to the phase-retrieval problem. Note that the equivalence
between the left- and right-hand side of \eqref{eq:rankmin} is
straightforward since by definition, there exists a rank-one solution.
Therefore, our problem is a rank minimization problem over an affine
slice of the positive semidefinite cone. As such, it falls in the
realm of low-rank {\em matrix completion} or {\em matrix recovery}, a
class of optimization problems that has gained tremendous attention in
recent years. Just as in matrix
completion, the linear system $\cA(X) = b$, with unknown in the
positive semidefinite cone, is highly underdetermined. For instance
suppose our signal $x_0$ has $n$ complex unknowns. Then we may imagine
collecting six diffraction patterns with $n$ measurements for each (no
oversampling). Thus $m = 6n$ whereas the dimension of the space of $n
\times n$ Hermitian matrices over the reals is $n^2$, which is
obviously much larger.


\subsection{Recovery via convex programming}
The rank minimization problem \eqref{eq:rankmin} is NP hard. 
We propose using the trace norm as a convex surrogate
for the rank functional, giving the familiar
SDP (and a crucial component of PhaseLift),
\begin{equation}
\label{eq:tracemin}
 \begin{array}{ll}
    \text{minimize}   & \quad \text{trace}(X)\\ 
    \text{subject to} & \quad  \cA(X) = b\\
& \quad X \succeq 0. 
\end{array}
\end{equation}

In our implementation, we propose to solve the following convex optimization problem in CVX. Here, $\lambda$ can be set to zero.

\begin{equation}
  \label{eq:tracemin}
   \begin{array}{ll}
      \text{minimize}   & \quad \| \cA(X) - b\|+\lambda \|X\|_1\\ 
  & \quad X \succeq 0. 
  \end{array}
  \end{equation}


\section{Numerical Experiments}
\label{numsec}

We present some numerical experiments to assess the empirical
performance of the Wirtinger flow algorithm. Here, we mostly consider a model of
coded diffraction patterns reviewed below.


\subsection{The coded diffraction model}
\label{defCDP}

We consider an acquisition model, where we collect data of the form  
\begin{equation}
\label{mainCDPmodel}
  y_r = \left| \sum_{t = 0}^{n-1} x[t] \bar{d}_\ell(t) e^{-i2\pi k t/n}
  \right|^2, \quad r = (\ell, k), \quad \begin{array}{l} 0 \le k \le n-1\\
    1 \le \ell \le L
  \end{array};
  \end{equation}
  thus for a fixed $\ell$, we collect the magnitude of the diffraction
  pattern of the signal $\{x(t)\}$ modulated by the waveform/code
  $\{d_\ell(t)\}$. By varying $\ell$ and changing the modulation
  pattern $d_\ell$, we generate several views thereby creating a
  series of \emph{coded diffraction patterns} (CDPs).

  In this report, we are mostly interested in the situation where the
  modulation patterns are random; in particular, we study a model in
  which the $d_\ell$'s are i.i.d.~distributed, each having
  i.i.d.~entries sampled from a distribution $d$.
\begin{align}
\label{momentcond}
\E d = 0, \quad \E d^2 = 0,\quad \E \abs{d}^4 =2{(\E \abs{d}^2)}^2.
\end{align}
A random variable obeying these assumptions is said to be {\em
  admissible}.  Since $d$ is complex valued we can have $\E d^2 = 0$
while $d \neq 0$. An example of an admissible random variable is $d =
b_1 b_2$, where $b_1$ and $b_2$ are independent and distributed as
\begin{align}
\label{eq:octanary}
b_1=\begin{cases}
 +1&\text{with prob.~} {1}/{4}\\
-1&\text{with prob.~} {1}/{4}\\
-i&\text{with prob.~} {1}/{4}\\
+i&\text{with prob.~} {1}/{4}
\end{cases}
\quad \text{and} \quad b_2=\begin{cases}
  \sqrt{2}/2 & \text{with prob.~}  {4}/{5}\\
\sqrt{3} & \text{with prob.~}  {1}/{5}
\end{cases}.
\end{align}
We shall refer to this distribution as an {\em octanary} pattern since
$d$ can take on eight distinct values.  The condition $\E[d^2]=0$ is
here to avoid unnecessarily complicated calculations in our
theoretical analysis.  In particular, we can also work with a {\em
  ternary} pattern in which $d$ is distributed as
\begin{equation}
\label{eq:ternary}
d=\begin{cases}
+1 &\quad\text{with prob.~} {1}/{4}\\
0  &\quad\text{with prob.~} {1}/{2}\\
-1 &\quad\text{with prob.~} {1}/{4}
\end{cases}. 
\end{equation}




\subsection{The Gaussian and coded diffraction models}

We begin by examining the performance of the Wirtinger flow algorithm for
recovering random signals $\vct{x}\in\C^n$ under the Gaussian and
coded diffraction models. We are interested in signals of two
different types:
\begin{itemize}
\item \emph{Random low-pass signals.} Here, $\x$ is given by 
  \begin{align*} {x}[t]=\sum_{k=-(M/2-1)}^{M/2} (X_k+iY_k) e^{2\pi i
      (k-1)(t-1)/n},
\end{align*}
with $M = n/8$ and $X_k$ and $Y_k$ are i.i.d.~$\mathcal{N}(0,1)$.
\item \emph{Random Gaussian signals.} In this model, $\x \in \C^n$ is
  a random complex Gaussian vector with i.i.d.~entries of the form
  $x[t] = X+iY$ with $X$ and $Y$ distributed as $\mathcal{N}(0,1)$;
  this can be expressed as
  \begin{align*} {x}[t]=\sum_{k=-(n/2-1)}^{n/2} (X_k+iY_k) e^{2\pi i
      (k-1)(t-1)/n},
\end{align*}
where $X_k$ and $Y_k$ are are i.i.d.~$\mathcal{N}(0,1/8)$ so that the
low-pass model is a `bandlimited' version of this high-pass random
model (variances are adjusted so that the expected signal power is the
same).
\end{itemize}
Below, we set $n=128$, and generate one signal of each type which will
be used in all the experiments.

The initialization step of the Wirtinger flow algorithm is run by applying $50$
iterations of the power method. In the iteration \eqref{graddescent}, we use the
parameter value $\mu_\tau=\min(1-\exp(-\tau/\tau_0),0.2)$ where
$\tau_0 \approx 330$. We stop after $2,500$ iterations, and report the
empirical probability of success for the two different signal
models. The empirical probability of succcess is an average over $100$
trials, where in each instance, we generate new random sampling
vectors according to the Gaussian or CDP models. We declare a trial
successful if the relative error of the reconstruction
$\operatorname{dist}(\hat{\vct{x}},\vct{x})/\twonorm{\vct{x}}$ falls
below $10^{-5}$, where we define $\emph{dist}(\vct{z},\vct{x}) = \min_{\phi\in[0,2\pi]}
\,\,\, \twonorm{\vct{z}-e^{i\phi}\vct{x}}$.

\subsection{Results presentation}
{\bf Wirtinger Flow}

In this part of the experiments, we set $n=32, 64, 128, 256$. 

In order to investigte the best choice of $L = m/n$, we perform a grid search, and evaluate 100 times to estimate the probability of success. A retrieval is called successful if $\|y-|Az_k|^2\|_2/\|y\| < 10^{-5}$, where $z_k$ is the retrieval vector. We use MATLAB file "test\_L.m" to generate the Figure \ref{fig: L32}, \ref{fig: L64}, \ref{fig: L128}, and \ref{fig: L256}. We find that for different $n$, similar $L$s are reuqired to guarantee reconstruction.

Then, we set $n=128$. In the Gaussian model, we use $L=4.5$ and in the coded diffusion model, we use $L=6$, which is presented in Figure \ref{fig: err}.


\begin{figure}[htbp]
  \centering
  \includegraphics[width=0.8\textwidth]{32.png}
  \caption{Probability of success under different $L$, $n=32$}
  \label{fig: L32}
\end{figure}

\begin{figure}[htbp]
    \centering
    \includegraphics[width=0.8\textwidth]{64.png}
    \caption{Probability of success under different $L$, $n=64$}
    \label{fig: L64}
\end{figure}

\begin{figure}[htbp]
  \centering
  \includegraphics[width=0.8\textwidth]{128.png}
  \caption{Probability of success under different $L$, $n=128$}
  \label{fig: L128}
\end{figure}

\begin{figure}[htbp]
  \centering
  \includegraphics[width=0.8\textwidth]{256.png}
  \caption{Probability of success under different $L$, $n=256$}
  \label{fig: L256}
\end{figure}

\begin{figure}[htbp]
    \centering
    \subfigure[Gaussian model]{
        \resizebox{!}{0.35\textwidth}{
        \input{test2.tex}}}
    \subfigure[Coded diffusion models]{
        \resizebox{!}{0.35\textwidth}{
        \input{test3.tex}}}
    \caption{Relative error-iteration}
    \label{fig: err}
\end{figure}


{\bf PhaseLift}

We find that CVX-based PhaseLift cannot solve large-scale problem when $n=128$. Hence, we only test its performance on small scale problems. However, Wirtinger flow still significantly outperforms the CVX-based PhaseLift. 

We take $n=20$ for example, and $L$ is set to be $6$ in the CDF case. The results can be summarized below, the error is defined by $\operatorname{dist}(\hat{\vct{x}},\vct{x})/\twonorm{\vct{x}}$
\begin{table}
  \centering
  \begin{tabular}{ccc}
    \toprule
    algorithm & time & error\\
    \midrule
    Wirtinger flow & 0.1276 & 2.7944e-16\\
CVX-based PhaseLift & 0.9412 &  1.7403e-11\\
    \toprule
  \end{tabular}
\end{table}



\bibliographystyle{apalike}
\bibliography{bib.bib}
\end{document}