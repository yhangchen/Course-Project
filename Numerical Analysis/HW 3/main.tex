\documentclass[conference,onecolumn, 12pt]{IEEEtran}
\IEEEoverridecommandlockouts
% The preceding line is only needed to identify funding in the first footnote. If that is unneeded, please comment it out.
\usepackage{cite}
\usepackage{amsmath,amssymb,amsfonts}
\usepackage{algorithm}
\usepackage{algorithmicx}
\usepackage{algpseudocode}
\usepackage{graphicx}
\usepackage{listings}
\usepackage{subfigure}
\usepackage{textcomp}
\usepackage{xcolor}
\usepackage{fancyhdr}
\usepackage{booktabs}
\usepackage{multirow}
\usepackage{pgfplots}
\pgfplotsset{compat=newest}
%% the following commands are needed for some matlab2tikz features
\usetikzlibrary{plotmarks}
\usetikzlibrary{arrows.meta}
\usepgfplotslibrary{patchplots}
\usepackage{grffile}

\usepackage[colorlinks,linkcolor=blue,anchorcolor=blue,citecolor=blue]{hyperref}
\counterwithin{figure}{section}
\counterwithin{table}{section}
\counterwithin{equation}{section}
\newcommand{\f}{\mathbf{f}}
\newcommand{\x}{\mathbf{x}}
\usepackage{setspace}
\renewcommand{\baselinestretch}{1.5}

\def\BibTeX{{\rm B\kern-.05em{\sc i\kern-.025em b}\kern-.08em
    T\kern-.1667em\lower.7ex\hbox{E}\kern-.125emX}}
\begin{document}
\title{Gauss Quadrature in Triangles\\
{\footnotesize \textsuperscript{*}Report 3 on the course ``Numerical Analysis".}
}

\author{\IEEEauthorblockN{1\textsuperscript{st} Chen Yihang}
\textit{Peking University}\\
1700010780\\
}

\maketitle
\thispagestyle{fancy} % IEEE模板在\maketitle后会自动声明\thispagestyle{plain},
% 导致第一页什么都没有。所以得把plain更改为fancy
\lhead{} % 页眉左,需要东西的话就在{}内添加
\chead{} % 页眉中
\rhead{} % 页眉右
\lfoot{} % 页眉左
\cfoot{} % 页眉中
\cfoot{\thepage} %页眉右,\thepage 表示当前页码
\renewcommand{\headrulewidth}{0pt} %改为0pt即可去掉页眉下面的横线
\renewcommand{\footrulewidth}{1pt} %改为0pt即可去掉页脚上面的横线
\pagestyle{fancy}
\cfoot{\thepage}

\begin{abstract}
    This report investigates the Gauss quadrature rules in a triangle and tetrahedron. We use symmetry to reduce the number of equations and variables. Finally, we compare the difference between Gauss quadrature in one or higher dimensions. Also, we discuss the vanilla Newton algorithm and Broyden algorithm to solve the nonlinear equations w.r.t. the weights and coordinates of nodes.
\end{abstract}

\tableofcontents
\clearpage
\section{Nonlinear System of Equations}

\subsection{Newton iteration}
Assume $f\in \mathbb{R}^n\to \mathbb{R}^n$, we want to solve $f(x)=0$. The following recursive relation is adopted to find the roots
\begin{equation}
    x^{(k+1)}=x^{(k)}-\nabla f(x^{(k)})^{-1}f(x^{(k)})
\end{equation}

In computing $\nabla f(x^{(k)})^{-1}f(x^{(k)})$, we can either calculate $\nabla f(x^{(k)})$ and solve the linear equation in each time, or use Broyden algorithm
\begin{subequations}
    \begin{align}
        &(A^{(0)})^{-1}=\nabla f(x^{(0)})^{-1}\\
        \begin{split}
            (A^{(k)})^{-1} - (A^{(k-1)})^{-1} =
            -\frac{[(A^{(k-1)})^{-1}g^{(k-1)}-y^{(k-1)}](y^{(k-1)})^\top (A^{(k-1)})^{-1}}{(y^{(k-1)})^\top(A^{(k-1)})^{-1}g^{(k-1)}}
        \end{split}
    \end{align}
\end{subequations}
where
\begin{subequations}
    \begin{align}
        g^{(k-1)}&=f(x^{(k)})-f(x^{(k-1)})\\
        y^{(k-1)}&=x^{(k)}-x^{(k-1)}
    \end{align}
\end{subequations}

In our implementation, we use variable ``flag'' to differentiate these two cases. 
\subsection{Homotopy method}
We use the homotopy function
\begin{equation}
    h(x,\lambda)=f(x)+(\lambda-1)f(x_0)
\end{equation}
and we take $\lambda_i=\frac{i}{n}$ in the $i$-th iteration. When $\lambda_i < 1$, we set the stopping rule to be $\|h(x,\lambda_i)\|_\infty<10^{-6}$. When $\lambda=1$, the stopping rule is set to be $\|f(x)\|_\infty<2^{-52}$.

\subsection{Comparison between two cases}
We use the objective function from ``test4.m'' to compare the performace between these two cases in ``test5.m''. We have the following table \ref{tab:comp}, we find that vanilla method is more stable than Broyden algorithm.
% Table generated by Excel2LaTeX from sheet 'Sheet1'
\begin{table}[htbp]
    \centering
    \caption{Comparison between two method}
      \begin{tabular}{lrrlrrr}
        \toprule
      \multicolumn{1}{c}{\multirow{2}[0]{*}{initial point}} & \multicolumn{2}{c}{$\ell_\infty$ error} & \multicolumn{2}{c}{iter} & \multicolumn{2}{c}{time} \\
      \cmidrule{2-3} \cmidrule{4-5}\cmidrule{6-7}
            & \multicolumn{1}{l}{Broyden} & \multicolumn{1}{l}{vanilla} & Broyden & \multicolumn{1}{l}{vanilla} & \multicolumn{1}{l}{Broyden} & \multicolumn{1}{l}{vanilla} \\
            \midrule
      $(0,0,0,0)^\top$ & 2.7756e-17 & 1.7347e-18 & \multicolumn{1}{r}{58} & 27    & 0.0135 & 0.0067 \\
      $(1,0,0,0)^\top$ & 6.9389e-18 & 3.4694e-18 & \multicolumn{1}{r}{301} & 36    & 0.0149 & 0.0087 \\
      $(1,1,0,0)^\top$ & 1.2250e+25 & 3.4694e-18 & \multicolumn{1}{r}{max} & 36    & 0.0922 & 0.0068 \\
      $(1,1,0,0)^\top$ & 5.7000e-03 & 3.4694e-18 & \multicolumn{1}{r}{max} & 57    & 0.0815 & 0.0073 \\
      $(1,1,1,1)^\top$ & 8.5001e-17 & 5.2042e-18 & \multicolumn{1}{r}{168} & 36    & 0.0147 & 0.0061 \\
      $(0.5,0.5,0.5,0.5)^\top$ & $\infty$ & 5.2042e-18 & \multicolumn{1}{r}{max}   & 231   & 0.0874 & 0.0171 \\
      \bottomrule
      \end{tabular}%
    \label{tab:comp}%
  \end{table}%
  
Then, we explore the trajectory of infinity error, we set the initial point to be $(0,0,0,0)^\top$, and vary the number of homotopy classes. The results are plotted in Figure \ref{fig:comp1} and \ref{fig:comp2}.
\begin{figure}[!htbp]
    \centering
    \subfigure[num of homotopy classes $=10$]{
    \resizebox{0.8\textwidth}{!}{
        \input{fig1.tex}
    }
    }
    \subfigure[num of homotopy classes $=5$]{
    \resizebox{0.8\textwidth}{!}{
        \input{fig2.tex}
    }
    }
    \subfigure[num of homotopy classes$=2$]{
    \resizebox{0.8\textwidth}{!}{
        \input{fig3.tex}
    }
    }
    \caption{$\ell_\infty$ error of homotopy function $H(x,\lambda)$}
    \label{fig:comp1}
\end{figure}

\begin{figure}[!htbp]
    \centering
    \subfigure[num of homotopy classes $=10$]{
    \resizebox{0.8\textwidth}{!}{
        \input{1fig1.tex}
    }
    }
    \subfigure[num of homotopy classes $=5$]{
    \resizebox{0.8\textwidth}{!}{
        \input{1fig2.tex}
    }
    }
    \subfigure[num of homotopy classes$=2$]{
    \resizebox{0.8\textwidth}{!}{
        \input{1fig3.tex}
    }
    }
    \caption{$\ell_\infty$ error of objective function $f(x)$}
    \label{fig:comp2}
\end{figure}


We find that more homotopy classes are crucial for convergence, especially for Broyden algorithm, which fails to converge when the number of classes is 2 but is able to converge when the number of classes is 5. Generally, vanilla Newton iteration requires fewer iterations than Broyden algorithm, while requires more computational budgets from solving the linear equations.

Then, we compare the Figure \ref{fig:comp1} and \ref{fig:comp2}, we find that solving $H(x,\lambda)=0$ for $\lambda\neq 1$ only slowly decreases the error of the residual, but is crucial for convergence. The order of convergence in the neighborhood of optimal point is second order.
\section{Triangle}
We can use a linear transformation to transform any triangle into the standard triangle, whose vertices are $(0,0), (0,1), (1,0)$. Assume the vertices of the original triangle is $(x_i,y_i),1\leq i\leq 3$, the linear transformation is 
\begin{equation}
    \begin{split}
        x(s,t) &= x_1+(x_2-x_1)s+(x_3-x_1)t\\
        y(s,t) &= y_1+(y_2-y_1)s+(y_3-y_1)t\\
    \end{split}
\end{equation}

Hence, $\frac{\partial(x,y)}{\partial(s,t)}=2Area$, and 
\begin{equation}
    I(f) = 2Area \int_{s=0}^1\int_{t=0}^{1-s} f(x(s,t),y(s,t)) dtds
\end{equation}
Thus, we need only compute nodes and weights on a standard triangle.
\subsection{Polynomials of order 1}
We only use one node.
\begin{subequations}
    \begin{align}
        f(s,t)=1&\quad w_1=\frac{1}{2}\\
        f(s,t)=s&\quad w_1x_1=\frac{1}{6}\\
        f(s,t)=t&\quad w_1y_1=\frac{1}{6}
    \end{align}
\end{subequations}

To sum up, we have the node is $(\frac{1}{3},\frac{1}{3})$ and weight is $\frac{1}{2}$. We plot our results in figure \ref{p1}.
\begin{figure}[!htbp]
    \centering
    \includegraphics[width=3in,height=3in]{order1.eps}
    \caption{Polynomials of order 1}
    \label{p1}
\end{figure}

\subsection{Polynomials of order 2}
We first shows that two points are not enough. Otherwise, 
\begin{subequations}
    \label{twopoints}
    \begin{align}
        f(s,t)=1&\quad w_1+w_2=\frac{1}{2}\\
        f(s,t)=s&\quad w_1x_1+w_2x_2=\frac{1}{6}\\
        f(s,t)=t&\quad w_1y_1+w_2y_2=\frac{1}{6}\\
        f(s,t)=s^2&\quad w_1x_1^2+w_2x_2^2=\frac{1}{12}\\
        f(s,t)=t^2&\quad w_1y_1^2+w_2y_2^2=\frac{1}{12}\\
        f(s,t)=st&\quad w_1x_1y_1+w_2x_2y_2=\frac{1}{24}
    \end{align}
\end{subequations}
% whose Jacobian is
% \begin{equation}
%     J_1 = \left(
%         \begin{matrix}
%             1& 1&0&0&0&0\\
%             x_1&x_2&w_1&w_2&0&0\\
%             y_1&y_2&0&0&w_1&w_2\\
%             x_1^2&x_2^2&2w_1x_1&2w_2x_2&0&0\\
%             y_1^2&y_2^2&0&0&2w_1y_1&2w_2y_2\\
%             x_1y_1&x_2y_2&w_1y_1&w_2y_2&w_1x_1&w_2x_2\\
%         \end{matrix}
%     \right)
% \end{equation}
% and $\det(J_1)=0$, which is easy to be verified by the following code.
% \begin{lstlisting}[language=matlab]
%     syms w1 w2 x1 x2 y1 y2
%     f = [w1+w2-1;
%     w1*x1+w2*x2-1/6;
%     w1*y1+w2*y2-1/6;
%     w1*x1^2+w2*x2^2-1/12;
%     w1*y1^2+w2*y2^2-1/12;
%     w1*x1*y1+w2*x2*y2-1/24];
%     df = jacobian(f,[w1 w2 x1 x2 y1 y2]);
%     det(df)
% \end{lstlisting}

We prove that the function have no solution.
\begin{itemize}
    \item {\bf Proof}:\\
    Assume the solution exist, then we have 
    \begin{equation}
        \begin{split}
            f(s,t)&=w_1(x_1s+y_1t+1)^2+w_2(x_2s+y_2t+1)^2\\
            &=\frac{1}{12}(s^2+t^2+st)+\frac{1}{3}(s+t)+\frac{1}{2}\\
            &=\frac{1}{12}((s+\frac{t}{2}+2)^2+\frac{3}{4}(t+\frac{4}{3})^2+\frac{2}{3})>0
        \end{split}
    \end{equation}
    Besides, we $x_1y_2\neq x_2y_1$, otherwise, $$(w_1x_1^2+w_2x_2^2)(w_1y_1^2+w_2y_2^2)=( w_1x_1y_1+w_2x_2y_2)^2$$
    which is contrary to equations \ref{twopoints}. Thus, there exist $s,t$, such that$$x_1s+y_1t+1=x_2s+y_2t+1=0$$which leads to $f(s,t)=0$, contrary to the fact that $f(s,t)>0$.
    In sequel, the set of equations has no real roots. There are at least three node points.
\end{itemize}

Then, when there are three points, we have
\begin{subequations}
    \label{threeeq}
    \begin{align}
        f(s,t)=1&\quad w_1+w_2+w_3=\frac{1}{2}\label{threeeq1}\\
        f(s,t)=s&\quad w_1x_1+w_2x_2+w_3x_3=\frac{1}{6}\label{threeeq2}\\
        f(s,t)=t&\quad w_1y_1+w_2y_2+w_3y_3=\frac{1}{6}\label{threeeq3}\\
        f(s,t)=s^2&\quad w_1x_1^2+w_2x_2^2+w_3x_3^2=\frac{1}{12}\label{threeeq4}\\
        f(s,t)=t^2&\quad w_1y_1^2+w_2y_2^2+w_3y_3^2=\frac{1}{12}\label{threeeq5}\\
        f(s,t)=st&\quad w_1x_1y_1+w_2x_2y_2+w_3x_3y_3=\frac{1}{24}\label{threeeq6}
    \end{align}
\end{subequations}

Assuming $x_1=0,y_2=0,x_3+y_3=1$, we have
\begin{subequations}
\begin{align}
    w_1+w_2+w_3=\frac{1}{2}\\
    w_2x_2+w_3x_3=\frac{1}{6}\\
w_1y_1+w_3(1-x_3)=\frac{1}{6}\\
w_2x_2^2+w_3x_3^2=\frac{1}{12}\\
w_1y_1^2+w_2y_2^2+w_3y_3^2=\frac{1}{12}\\
w_3x_3(1-x_3)=\frac{1}{24}
\end{align}    
\end{subequations}
The solution can clearly be obtained by $w_i=\frac{1}{6},t=\frac{1}{2}$. The code is implmented in ``test1.m'', where we take initial value $x^{(0)}=x^\star+\varepsilon N(0,I)$, $x^\star=(\frac{1}{6},\frac{1}{6},\frac{1}{6},\frac{1}{2},\frac{1}{2},\frac{1}{2})$.

Another approach, which is more inspiring, is by symmetry. Clearly, we should anticipate that $w_1=w_2=w_3:=w$, and
\begin{equation}
    \begin{split}
        (x_2,y_2)=(1,0)+y_1(-1,0)+x_1(-1,1)\\
        (x_3,y_3)=(0,1)+y_1(1,-1)+x_1(0,-1)    
    \end{split}
\end{equation}

Clearly, from equation \ref{threeeq1}, $w=\frac{1}{6}$. Hennce, only two variables remains. Equation \ref{threeeq2}, \ref{threeeq3} hold, and equation \ref{threeeq4}, \ref{threeeq5}, \ref{threeeq6} are equivalent. Hence, we have
\begin{equation}
        x_1^2+(1-x_1-y_1)^2+y_1^2 = \frac{1}{2}\\
\end{equation}
which constitutes an ellipse inscribes the triangle.

\begin{figure}[!htbp]
    \centering
    \includegraphics[width=3in,height=3in]{order2.eps}
    \caption{Polynomials of order 2}
    \label{p2}
\end{figure}
Codes to generate the figure \ref{p2}:
\begin{lstlisting}[language=matlab]
    patch([0,1,0],[0,0,1],[1,1,1])
    fimplicit(@(x,y) x.^2+y.^2+(1-x-y).^2-1/2)
    scatter(0.0931,0.2533,'fill','k')
    scatter(1-0.0931-0.2533,0.0931,'fill','k')
    scatter(0.2533,1-0.0931-0.2533,'fill','k')
\end{lstlisting}

Hence, there are infinite nodes choices, $(s,t),(1-s-t,s),(t,1-s-t)$, where $s^2+t^2+(1-s-t)^2=\frac{1}{2}$, and unified weights $w_i=\frac{1}{6}$.



\subsection{Polynomials of order 3}
Since there are 10 equations, i.e. $1,x,y,x^2,xy,y^2,x^3,x^2y,\\xy^2,y^3$ in total, at least 4 node points are required, leading to 12 variables. By symmetry, we can assume one of the node is $(\frac{1}{3},\frac{1}{3})$ with weight $w_2$, and the other three node points are in the form $(s,t),(1-s-t,s),(t,1-s-t)$, with unified weight $w_1$. We thus have $3w_1+w_2=\frac{1}{2}$ by setting $f(x,y)=1$ or $x$.

% which leads to $w_1=\frac{25}{96},w_2=-\frac{27}{96}$.

Hence, we have
\begin{subequations}
    \label{fourpoints}
    \begin{align}
        3w_1+w_2=\frac{1}{2}\label{1dhwbd1}\\ 
        w_1(s^2+t^2+(1-s-t)^2)+w_2\frac{1}{9}=\frac{1}{12}\label{1dhwbd}\\
        w_1(s^2t+(1-s-t)^2s+s^2(1-s-t))+w_2\frac{1}{9}=\frac{1}{60}\\
        w_2(s^3+t^3+(1-s-t)^3)+w_2\frac{1}{27}=\frac{1}{20}
    \end{align}
\end{subequations}

Another criterion
\begin{equation}
    w_1(st+(1-s-t)(s+t))+w_2\frac{1}{9}=\frac{1}{24}
\end{equation}
is equivalent to \ref{1dhwbd1} and \ref{1dhwbd}.

We use ``test2.m'' to compute the solution of \ref{fourpoints}. The initial point is set to be $\mathbf{x}^{(0)}=(0,0,0) +\varepsilon N(0,I)$. The obtained solution is
\begin{equation}
    \label{eiebvi}
    s=t=1/5,w_2=-\frac{9}{32}
\end{equation}

\begin{figure}[!htbp]
    \centering
    \includegraphics[width=3in,height=3in]{order3.eps}
    \caption{Polynomials of order 3}
    \label{p3}
\end{figure}


From other initial points, we are able to obtain other solutions. However, they are either equivalent to the solution \ref{eiebvi} or outside the reference triangle, which is not preferrable.
% We find that if even though there are 3 solutions, $x^\star_1=(\frac{1}{5},\frac{1}{5}),x^\star_2=(\frac{1}{5},\frac{3}{5}),x^\star_3=(\frac{3}{5},\frac{1}{5})$. Our algorithms tend to converge to $x^\star_2$ and $x^\star_3$ instead of $x^\star_1$, despite their theoretical symmetry.
% \begin{figure}[!htbp]
%     \centering
%     \includegraphics[width=3in,height=3in]{order3.eps}
%     \caption{Polynomials of order 3}
%     \label{p3}
% \end{figure}
% By setting $x^{(0)}=x^\star_1+\varepsilon N(0,I)$, we find that $x^\star_2$ and $x^\star_3$ allows larger $\varepsilon$, such as 0.1. However, when we set $i=1$, setting $\varepsilon=0.1$ renders divergence, larger $\varepsilon$ will leads the trajectory into $x^\star_2$ or $x^\star_3$. Only if we set $\varepsilon=0.005$, the trajectory will converge to $x^\star_1$.  

% We plot figures of $\|f\|_2^2$ to show the difference of local surface around $x^\star_1$ and $x^\star_2$. Owing to linear transformation, the ``valley'' around $x^\star_1$ is narrower than $x^\star_2$ and $x^\star_3$. 
% \begin{figure}[!htbp]
%     \centering
%     \subfigure[Local surface around $x^\star_1$]{
%         \includegraphics[width=3in,height=3in]{sol1.eps}
%     }
%     \subfigure[Local surface around $x^\star_2$]{
%         \includegraphics[width=3in,height=3in]{sol2.eps}
%     }
%     \subfigure[Contour]{
%         \includegraphics[width=3in,height=2.5in]{contour.eps}
%     }
%     \caption{Local loss surface}
% \end{figure}

\subsection{Results on a equilateral triangle}
\paragraph{Polynomials of order 1}
The node is $\left(\frac{1}{2},\frac{\sqrt{3}}{6}\right)$, and the weight is $\frac{\sqrt{3}}{4}$.
\paragraph{Polynomials of order 2}
The nodes are on the circle $C = \{\left(x-\frac{1}{2}\right)^2+\left(y-\frac{\sqrt{3}}{6}\right)^2=\frac{1}{12}\}$, being the vertices of the inscribed equilateral triangle of the circle $C$, with weight $\frac{\sqrt{3}}{12}$.
\paragraph{Polynomials of order 3}
The nodes are $\left(\frac{1}{2},\frac{\sqrt{3}}{6}\right)$ with weight $-\frac{27\sqrt{3}}{192}$, $\left(\frac{3}{10},\frac{\sqrt{3}}{10}\right),\left(\frac{7}{10},\frac{\sqrt{3}}{10}\right),\left(\frac{1}{2},\frac{3\sqrt{3}}{10}\right)$ with weight $\frac{25\sqrt{3}}{192}$.

\section{Tetrahedron}
We can use a linear transformation to transform any tetrahedron into the standard one, whose vertices are $(0,0,0), (1,0,0), (0,1,0), (0,0,1)$. Assume the vertices of the original triangle is $(x_i,y_i,z_i),1\leq i\leq 4$, the linear transformation is 
\begin{equation}
    \label{3dimtrans}
    \begin{split}
        x(u,v,w) &= x_1+(x_2-x_1)u+(x_3-x_1)v+(x_4-x_1)w\\
        y(u,v,w) &= y_1+(y_2-y_1)u+(y_3-y_1)v+(y_4-y_1)w\\
        z(u,v,w)&= z_1+(z_2-z_1)u+(z_3-z_1)v+(z_4-z_1)w\\
    \end{split}
\end{equation}

Hence, $\frac{\partial(x,y,z)}{\partial(u,v,w)}=6Volume$, and 
\begin{equation}
    \begin{split}
        I(f) &= 6\text{Volume} \int_{u=0}^1\int_{v=0}^{1-u}\int_{w=0}^{1-u-v} \\
        &f(x(u,v,w),y(u,v,w),z(u,v,w)) dudvdw
    \end{split}
\end{equation}
Thus, we need only compute nodes and weights on a standard tetrahedron, then the nodes can be transformed according to \ref{3dimtrans}, and the weights $w$ can be transformed to $6\text{Volume}\times w$. We plot the standard tetrahedron in the following code
\begin{lstlisting}[language=matlab]
    h = patch('Vertices',[0 0 0; 1 0 0;
     0 1 0; 0 0 1],'Faces',[1 3 2;1 2 4;
     1 3 4;2 3 4],'FaceColor',[1 1 1])
    view(3)
    alpha(0)
\end{lstlisting}
\subsection{Polynomials of order 1}
Setting $f=1$, we have $w_1=\frac{1}{6}$. Besides, setting $f(x,y,z)=x$, we have $w_1x_1=\frac{1}{24}$, then $x=\frac{1}{4}$. By symmetry, the node is $\left(\frac{1}{4},\frac{1}{4},\frac{1}{4}\right)$.
\begin{figure}[!htbp]
    \centering
    \includegraphics[width=4in]{3order1.eps}
    \caption{Polynomials of order 1, 3 dimension}
    \label{3p1}
\end{figure}
\subsection{Polynomials of order 2}
Since there are $x^2,y^2,z^2,xy,yz,xz,x,y,z,1$ (10 constraints), and each node provides us with 4 variables, we need at least three nodes. By symmetry, we need four nodes for symmetry. Since the weights are the same, we have the unified weight to be $w=\frac{1}{24}$. The nodes should be
\begin{equation}
    \mathbf{x}_1=(x,y,z)^\top,\mathbf{x}_2=(1-x-y-z,x,y)^\top, \mathbf{x}_3=(z,1-x-y-z,x)^\top,\mathbf{x}_4=(y,z,1-x-y-z)^\top
\end{equation}
We have the following equation
\begin{subequations}
    \begin{align}
        x^2+y^2+z^2+(1-x-y-z)^2=\frac{2}{5}\\
        (1-x-z)(x+z)=\frac{1}{5}\\ 
        (1-y-z)(x+y)=\frac{1}{5}\\
        (1-x-y)(x+y)=\frac{1}{5}
    \end{align}
\end{subequations}
Note that adding the last three equations up, we get the first equation. We solve this equation in ``test3.m''. Clearly, the solution has multiple solutions. The results is
\begin{equation}
    \begin{split}
        \mathbf{x}_1=(0.5854,0.1382,0.1382)^\top,\mathbf{x}_2=(0.1382,0.5854,0.1382)^\top\\
        \mathbf{x}_3=(0.1382,0.1382,0.5854)^\top,\mathbf{x}_4=(0.1382,0.1382,0.1382)^\top
    \end{split}
\end{equation}
if we restrict the nodes to be inside the tetrahedron. The exact solution is
$x=y=z=\frac{5-\sqrt{5}}{10}$.

The program produces other results, 
\begin{equation}
    \begin{split}
        \mathbf{x}_1=(0.3618,0.3618,0.3618)^\top,\mathbf{x}_2=(-0.0854,0.3618,0.3618)^\top\\
        \mathbf{x}_3=(0.3618,-0.0854,0.3618)^\top,\mathbf{x}_4=(0.3618,0.3618,-0.0854)^\top
    \end{split}
\end{equation}
which is outside the tetrahedron.
\begin{figure}[!htbp]
    \centering
    \includegraphics[width=4in]{3order2.eps}
    \caption{Polynomials of order 2, 3 dimension}
    \label{3p2}
\end{figure}
\subsection{Polynomials of order 3}
If there are only 4 points, we have 
\begin{equation*}
    (0.1382^3+0.1382^3+0.1382^3+0.5854^3)/24\neq \frac{1}{120}=\int_{\Delta} x^3 {\rm d}V
\end{equation*}
which means 4 points are not enough. Hence, at least 5 points are required. Still, by symmetry, we assume one of the points is $(\frac{1}{4},\frac{1}{4},\frac{1}{4})^\top$ with weight $w_2$ and the other three points
\begin{equation}
    \mathbf{x}_1=(x,y,z)^\top,\mathbf{x}_2=(1-x-y-z,x,y)^\top, \mathbf{x}_3=(z,1-x-y-z,x)^\top,\mathbf{x}_4=(y,z,1-x-y-z)^\top
\end{equation}
with weight $w_1$. Hence, by setting the degree of function to be zero or one, we have
\begin{equation}
4w_1+w_2=\frac{1}{6}
\end{equation}
Besides, by setting the degree of function to be 2.
\begin{subequations}
    \begin{align}
        w_1(x^2+y^2+z^2+(1-x-y-z)^2)+w_2\frac{1}{16}=\frac{1}{60}\label{n3d31}\\
        w_1(1-x-z)(x+z)+w_2\frac{1}{16}=\frac{1}{120}\\ 
        w_1(1-y-z)(x+y)+w_2\frac{1}{16}=\frac{1}{120}\\
        w_1(1-x-y)(x+y)+w_2\frac{1}{16}=\frac{1}{120}
    \end{align}
\end{subequations}

Since \ref{n3d31} can be derived from other equations, we use the following set of equations to obtain the results
\begin{subequations}
    \begin{align}
        4w_1+w_2=\frac{1}{6}\\
        w_1(1-x-z)(x+z)+w_2\frac{1}{16}=\frac{1}{120}\\ 
        w_1(1-y-z)(x+y)+w_2\frac{1}{16}=\frac{1}{120}\\
        w_1(1-x-y)(x+y)+w_2\frac{1}{16}=\frac{1}{120}\\
        w_1(x^3+y^3+z^3+(1-x-y-z)^3)1+w_2\frac{1}{64}=\frac{1}{120}
    \end{align}
\end{subequations}
which is implmented in ``test4.m''. The results are
\begin{equation}
    w_1=\frac{3}{40},w_2=-\frac{2}{15},\quad x=y=z=\frac{1}{6}
\end{equation}

To validate the solution indeed satisfies all polynomials whose degree is not larger than 3. We still need to verify (by symmetry, if $f(x,y,z)=x^2y$ is satisfied, then $f(x,y,z)=y^2x,y^2z,z^2y,x^2z,z^2x$ are also satisfied.)
\begin{equation}
    \begin{split}
        w_1(x^2y+(1-x-y-z)^2 x+z^2(1-x-y-z)+y^2z)+w_2\frac{1}{64}=\int_{\Delta} x^2y {\rm d}V\\
        w_1(xyz+(xy+yz+zx)(1-x-y-z))+w_2\frac{1}{64}=\int_{\Delta} xyz {\rm d}V
    \end{split}
\end{equation}
which are correct by direct computation
\begin{equation*}
    \int_{\Delta} x^2y {\rm d}V=\int_0^1 x^2 dx \int_0^{1-x} y dy \int_0^{1-x-y}dz = \int_0^1 x^2 dx \int_0^{1-x} (1-x-y) y dy=\frac{1}{360}
\end{equation*}
and
\begin{equation*}
    \begin{split}
        \int_{\Delta} xyz {\rm d}V=\int_0^1 x dx \int_0^{1-x} y dy \int_0^{1-x-y} zdz=\int_0^1 x dx \int_0^{1-x} \frac{1}{2}y(1-x-y)^2dy\\
        =  \frac{1}{24}  \int_0^1 x(1-x)^4 dx=\frac{1}{720}
    \end{split}
\end{equation*}
\begin{figure}[!htbp]
    \centering
    \includegraphics[width=4in]{3order3.eps}
    \caption{Polynomials of order 3, 3 dimension}
    \label{3p3}
\end{figure}


\section{Summary}
\subsection{Summary of codes}
\begin{itemize}
    \item {\bf newton\_homotopy}: implements the Newton's method to solve nonlinear equations with homotopy method.
    \item {\bf test1}: Find nodes and weights in a triangle second order algebraic precision.
    \item {\bf test2}: Find nodes and weights in a triangle third order algebraic precision.
    \item {\bf test3}: Find nodes and weights in a tetrahedron second order algebraic precision.
    \item {\bf test4}: Find nodes and weights in a tetrahedron second order algebraic precision.
    \item {\bf test5}: Comparison between vanilla Newton's iteration (flag = 0) as well as Broyden algorithm (flag = 1).


\end{itemize}
\subsection{Summary of experiments}
In all our experiments, the validity of solution is obtained by calculating the residual of the solution and verified by the fact that the scale of the residual is below the machine epsilon ($2^{-52}$). We have the following two observation.
\begin{enumerate}
    \item The nodes and corresponding weights might not be unique given a reference region. 
    \item The nodes could be outside the region and the weights could be negative.
\end{enumerate} 
which are completely different from one-dimensional Gauss quadrature rule, where nodes are uniquely inside the region and weights are guaranteed to be positive.
\end{document}

